%\chapter{Einleitung}
%\chapter{Introduction}
%\chapter{R�sum� und Ausblick}
\chapter{Conclusion}
\label{sec-conclusion}
The MOVE-II CubeSat to project is a project aiming to an educational platform and give students relevant experience in the space industry. To accomplice this a CubeSat has been built primarily by students. The satellite is expected to be launched some time in October. The payload of the satellite is experimental solar cells. To conduct recordings of the efficacy of the solar cell when they are in space it is required that they face the sun. To accomplish this a ADCS was needed. As part of the ADCS a attitude determination system based around a extended Kalman filter was design. This ADS was not sufficiently tested before the launch software was locked, but the software can still be updated when the satellite is in space. Resent simulations show that the current controller is not very power efficient and there is a deicer to do an update of the controller. For the new controller the attitude needs to be determined. So it is therefore important to properly test and verify to current ADS to facilitate for this new controller. 

Two different methods for testing the ADS was developed and used. One was a simpler method not testing only the algorithm is self and not the full system called Algorithm only. the other method tested almost the full system and was called Outdoor. Bot test where able to give valuable  information about the functionality of the ADS. The main findings show that the ADS works nicely under good conditions. That is in the sun and whit a spinning satellite. The test also reveled some weaknesses of the ADS. Namely that the ADS does not work whit a stationary satellite and that the estimated attitude is very vulnerable to disturbances on the sun sensor. The reason for why it does not work whit no sun and a static satellite is still unknown. It also showed that the calculated reference models are correct.             