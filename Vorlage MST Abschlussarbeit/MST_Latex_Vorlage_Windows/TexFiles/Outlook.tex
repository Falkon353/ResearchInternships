\chapter{Outlook}
\label{sec-outlook}
Moving forward there are two main areas that should be looked at. The first is investigating some of the anomalies discovered by the test. The main anomaly that needs to be future investigated is why the ADS does not work with a stationary satellite and no sun. As the anomaly is reproducible in simulations it seems like continuing to investigate the problem in the simulation environment is the beater option. As the simulation environment is a lot easier and faster to work whit when you are doing debugging. Once the problem is discovered and solved one could go back to the hardware test for verification. 

Smaller issues to be resolved are solving the discrepancy between the reference model and the measurements. For the magnetometer is seems like changing the parameters for the calibrations method to get higher scaling is the way to proceed. For the sun sensor the way is a bit more unclear and a lot more investigation is needed. Looking into the transformation between sensor and body-fixed frame might be a starting point. 

The second are is development of the so called UWE-3 approach. The UWE-3 approach as the Outdoor test has the advantage that they test almost the entire system. The advantage the UWE-3 approach has over the Outdoor test is that is can be conducted using a permanent setup inside. This allows for a more efficient setup as you do not have to bring inn all the equipment outside all the time. It also has the advantage that it is not dependent on the sun. Meaning test could be conducted all year around. To develop the approach a suitable light source most be acquired and the function to find the correct time based on the position of the light source most be developed.                     